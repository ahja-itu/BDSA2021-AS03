% https://da.overleaf.com/latex/templates/cs310-assignment-template/qrqpndrxpcht
%%%%%%%%%%%%%%%%% DO NOT CHANGE HERE %%%%%%%%%%%%%%%%%%%% {
\documentclass[12pt,letterpaper]{article}
\usepackage{fullpage}
\usepackage[top=2cm, bottom=4.5cm, left=2.5cm, right=2.5cm]{geometry}
\usepackage{amsmath,amsthm,amsfonts,amssymb,amscd}
\usepackage{lastpage}
\usepackage{enumerate}
\usepackage{fancyhdr}
\usepackage{mathrsfs}
\usepackage{xcolor}
\usepackage{graphicx}
\usepackage{listings}
\usepackage{hyperref}

\hypersetup{%
    colorlinks=true,
    linkcolor=blue,
    linkbordercolor={0 0 1}
}

\setlength{\parindent}{0.0in}
\setlength{\parskip}{0.05in}
%%%%%%%%%%%%%%%%%%%%%%%%%%%%%%%%%%%%%%%%%%%%%%%%%%%%%%%%%% }

%%%%%%%%%%%%%%%%%%%%%%%% CHANGE HERE %%%%%%%%%%%%%%%%%%%% {
\newcommand\course{Analysis, Design and Software Architecture}
\newcommand\semester{\today}
\newcommand\hwnumber{3}                 % <-- ASSIGNMENT #
\newcommand\NetIDa{Andreas Wachs Hjalager}           % <-- YOUR NAME
\newcommand\NetIDb{Student number: 19167}           % <-- STUDENT ID #
%%%%%%%%%%%%%%%%%%%%%%%%%%%%%%%%%%%%%%%%%%%%%%%%%%%%%%%%%% }

%%%%%%%%%%%%%%%%% DO NOT CHANGE HERE %%%%%%%%%%%%%%%%%%%% {
\pagestyle{fancyplain}
\headheight 35pt
\lhead{\NetIDa}
\lhead{\NetIDa\\\NetIDb}
\chead{}
\rhead{\course \\ \semester}
\lfoot{}
\cfoot{}
\rfoot{\small\thepage}
\headsep 1.5em
%%%%%%%%%%%%%%%%%%%%%%%%%%%%%%%%%%%%%%%%%%%%%%%%%%%%%%%%%% }

\begin{document}
% Document begin
\begin{center}
    \textbf{\Large Assignment \hwnumber}
\end{center}

\section{GitHub repository link}

You should be able to click a link right \href{https://github.com/andreaswachs/BDSA2021-AS03}{here}, 
otherwise the link will be displayed just below, for you to copy:

\lstinline{https://github.com/andreaswachs/BDSA2021-AS03}

\section{C\# Part}

Data structures:

\begin{lstlisting}
    IEnumerable<int>[] xs;

    int[] ys;
\end{lstlisting}

\subsection{Q1: Flatten the numbers in $xs$}

Answer:

\begin{lstlisting}
    xs.SelectMany(items => items).Select(item => item);
\end{lstlisting}


\subsection{Q2: Select numbers in $ys$ which are divisible by $7$ and greater than $42$} 

Answer:

\begin{lstlisting}
    ys.Where(n => n % 7 == 0 && n > 42).Select(n => n)
\end{lstlisting}

\subsection{Q3: Select numbers in $ys$ which are *leap years*}

Answer:

\begin{lstlisting}
    ys.Where(n => n > 1582 && (n % 400 == 0 || n % 4 == 0) && n % 100 != 0)
\end{lstlisting}
\end{document}

